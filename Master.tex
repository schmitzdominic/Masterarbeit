\documentclass[12pt]{scrreprt}

%% Allgemeine Hinweise
% 
% - Windows ben�tigt Pearl zum Kompilieren
% - Guter Editor "TexStudio"
% - Nicht vergessen jedes Tex Dokument auf ISO-8859-1 zu stellen
% - Kompilierreihenfolge:
%		1. Makeglossaries
%		2. Biber
%		3. PdfLaTeX
%		4. Interner PDF-Betrachter

%--------------------------------------------------------------------------
% LaTeX packages
%--------------------------------------------------------------------------
\usepackage[ngerman]{babel}		% Neue Deute Rechtschreibung
\usepackage{pdfpages}	% PDF Import
\usepackage{geometry}	% Seitenformatierung (links, rechts usw..)
\usepackage{url}	% Urls
\usepackage{ae}		% Bessere Fonts unterst�tzung
\usepackage[latin1]{inputenc}	% Lateinisch und wichtig f�r MAC
\usepackage[T1]{fontenc}	% Westeurop�ische Kodierung
\usepackage{lmodern}	% Neue deutsche Trennungsregeln, etc 
\usepackage{mathptmx}	% Schriftart (Times New Roman)
\usepackage{acronym}	% Acronyme
\usepackage[onehalfspacing]{setspace}	% Eineinhalb Zeilen Abstand
\usepackage{titlesec}	% Titel Formatierung
\usepackage{hyperref}	% Klickbare Links
\usepackage{minitoc}	% Mini Inhaltsverzeichnis nach jeder �berschrift
\usepackage[headsepline,footsepline]{scrpage2}	% Kopf und Fu�zeilen (Seitenzahl)
\usepackage{listings}	% Quellcode
\usepackage{color}	% Farben f�rs Listing
\usepackage{float}	% Bilder zentrieren
\usepackage[right]{eurosym}
\usepackage[most]{tcolorbox} % Definitionen
\usepackage[autostyle, german=quotes]{csquotes}
\usepackage[backend=biber, style=alphabetic, citestyle=alphabetic]{biblatex}
\usepackage{chngcntr}
\usepackage{caption}
\usepackage{amsmath}
\usepackage{cleveref} % Definitionen
\usepackage[nonumberlist,	% Keine Seitenzahlen anzeigen
						acronym,	% Ein Abk�rzungsverzeichnis erstellen
						toc,	% Eintr�ge im Inhaltsverzeichnis
						section]	% Im Inhaltsverzeichnis auf section-Ebene erscheinen
						{glossaries}	% Glossar

%-----------------------------------------------------------------------------
% Settings
% ********
% Anmerkung: Die meisten Kommandos wurden in der 07_Settings/Settings.tex
% 						Datei erstellt und sollten auch nur dort ver�ndert werden.
%
%-----------------------------------------------------------------------------
% !TEX root = Settings.tex

%################################################################## COMMANDS
%##################################################################
%################################################################## 

\newcommand{\backsl}{\verb|\|} % Backslash

\newcommand{\secline}{\vspace{-1.2em}	% Linie
	\par\noindent\rule{\textwidth}{0.4pt}} 

\newcommand{\emptysite}{\newpage	% Leere Seite
	\thispagestyle{empty}
	\section*{}}

% Glossarzusatz
\renewcommand{\glossarysection}[2][]{\chapter*{#1}}	

% Abbildungsverzeichnis
\newcommand{\abbildungsverzeichnis}{\newpage
	\phantomsection
	\addstarredchapter{\textbf{Abbildungsverzeichnis}}
	\listoffigures
}

% Tabellenverzeichnis
\newcommand{\tabellenverzeichnis}{\newpage
	\phantomsection
	\addstarredchapter{\textbf{Tabellenverzeichnis}}
	\listoftables
}

% Acronyme
\newcommand{\abkuerzungen}{
	% !TEX root = Acronym.tex
\newpage
\phantomsection
\addstarredchapter{\textbf{Abk�rzungsverzeichnis}}

\chapter*{\huge\textbf{Abk�rzungsverzeichnis}}
\begin{acronym}[Bash]
	\acro{BSP}{Beispiel}
\end{acronym}
}

% Glossar
\newcommand{\glossar}{
	\newpage
	\phantomsection	
	\addstarredchapter{\textbf{Glossar}} % F�gt "Glossar" zum Inhaltsverzeichnis hinzu		
	%\addcontentsline{toc}{chapter}{\textbf{Glossar}}	% F�gt "Glossar" zum Inhaltsverzeichnis hinzu
	\printglossary[style=altlist,title=Glossar]		% Glossar
	\newpage
}

% Heading Style setzen
\newcommand{\headstyle}{
	\pagestyle{scrheadings}	% Style
	\clearscrheadfoot
	\ofoot[\pagemark]{\pagemark}
	\ohead{\headmark}
	\automark[]{chapter}
	\setheadsepline{1pt}
	\setfootsepline{0pt}
}

% Heading Style setzen (Seitenzahl Rechts)
\newcommand{\headstyleright}{
	\pagestyle{scrheadings}	% Style
	\cfoot[]{}	% [plain-Seiten]{normale Seiten} 
	\ofoot[\pagemark]{\pagemark}	% Seitenzahl
	\setheadsepline{0pt}
	\setfootsepline{0pt}
}

\newcommand{\inhaltsverzeichnis}{
	\cleardoublepage\pdfbookmark{\contentsname}{toc}
	\tableofcontents
	\clearpage
	\newpage
}

\newcommand{\literaturverzeichnis}{
	\phantomsection
	\addcontentsline{toc}{chapter}{Literatur}	% �berschrift
	\printbibliography	
}

\newcommand{\anhangstart}{
	\newpage
	\pagenumbering{arabic}	% Arabische Ziffern
	
	\renewcommand\appendix{\par 
		\setcounter{section}{0}% 
		\setcounter{subsection}{0}% 
		\setcounter{figure}{0}% 
		
		\renewcommand\thesection{\Alph{section}}% 
		\renewcommand\thefigure{\Alph{section}\arabic{figure}}} 
	
	\numberwithin{table}{section} 
	\numberwithin{figure}{section} 
	
	
	\appendix 
	\pagestyle{empty}
	\section*{Allgemeine Erg�nzungen}
	
	\captionsetup{listof=false}
}

\newcommand{\roemischenummerierung}{
	\pagenumbering{Roman}
}

\newcommand{\arabischenummerierung}{
	\pagenumbering{arabic}
}

%################################################################## OTHER
%##################################################################
%################################################################## 

\titleformat{name=\chapter,numberless}[display]{\normalfont\huge\bfseries}{\titlerule}{-6,9ex}{}[\vspace{3ex}]
\titlespacing{name=\chapter,numberless}{0em}{1.0em}{0em}
\titleformat{\chapter}{\normalfont\huge\bfseries}{\thechapter}{18pt}{\Huge}[{\vspace{0,6ex}\titlerule[0.8pt]\vspace{3ex}}]
\titlespacing{\chapter}{0em}{-4.0em}{0em} % {left}{before}{after}[right]
\titleformat{\section}{\normalfont\LARGE\bfseries}{\thesection}{16pt}{\LARGE}
\titleformat{\subsection}{\normalfont\Large\bfseries}{\thesubsection}{14pt}{\Large}
\titleformat{\subsubsection}{\normalfont\large\bfseries}{\thesubsubsection}{14pt}{\large}

\addbibresource{04_Literatur/Literatur.bib}
\AtBeginDocument{\counterwithin{lstlisting}{section}}
\geometry{a4paper, left=30mm, right=25mm, bottom=25mm, top=25mm}	% Text Formatierung
% !TEX root = Glossar.tex
\makeglossaries

\newglossaryentry{venturecapital}{ 
	name=Venture-Capital, 
	description={ 
		Unter Venture Capital versteht man die Finanzierung eines nicht b�rsennotierten Unternehmens mit Eigenkapital. Dabei wird das Unternehmen in der Regel zum Zeitpunkt der Beteiligung privat gehalten. Entscheidend ist, dass das Eigentum am Unternehmen w�hrend der meisten Zeit der Beteiligung in privaten H�nden liegt \autocite{venture_capital}} 
}

\newglossaryentry{servantleader}{ 
	name=Servant Leader, 
	description={ 
		Ein Servant Leader liebt Menschen und m�chte ihnen helfen. Die Aufgabe eines Servant Leaders ist es daher, die Bed�rfnisse anderer zu ermitteln und zu versuchen, diese Bed�rfnisse zu befriedigen \autocite{servant_leadership}} 
} 

\newglossaryentry{iteration}{ 
	name=Iteration, 
	description={ 
		HIER BESCHREIBUNG!
	}
} 

\newglossaryentry{stakeholder}{ 
	name=Stakeholder, 
	description={ 
		HIER BESCHREIBUNG!
	}
}

\newglossaryentry{trialanderror}{ 
	name=Trial and Error, 
	description={ 
		HIER BESCHREIBUNG!
	}
}
	% Glossar IMPORT
\setlength{\parindent}{0pt}	% Keine Einr�ckung am Anfang eines Absatzes
\setcounter{biburllcpenalty}{7000}
\setcounter{biburlucpenalty}{8000}

%################################################################## LISTINGS
%##################################################################
%################################################################## 

\renewcommand{\lstlistingname}{Codebeispiel}% Listing -> Codebeispiel
\renewcommand{\lstlistlistingname}{Codeverzeichnis}% List of Listings -> Codeverzeichnis

% Default fixed font does not support bold face
\DeclareFixedFont{\ttb}{T1}{txtt}{bx}{n}{12} % for bold
\DeclareFixedFont{\ttm}{T1}{txtt}{m}{n}{12}  % for normal

% Custom colors
\definecolor{deepblue}{rgb}{0,0,0.5}
\definecolor{deepred}{rgb}{0.6,0,0}
\definecolor{deepgreen}{rgb}{0,0.5,0}
\definecolor{superlightgray}{rgb}{0.97,0.97,0.97}

% Java Style
\lstdefinestyle{java}{
	language=Java,
	basicstyle=\ttm,
	otherkeywords={this},
	breaklines=true,
	tabsize=2,
	keywordstyle=\ttb\color{deepblue},
	emph={MyClass,__init__},          % Custom highlighting
	emphstyle=\ttb\color{deepred},    % Custom highlighting style
	stringstyle=\color{deepgreen},
	frame=tb,                         % Any extra options here
	belowskip=3em,
	belowcaptionskip=1em,
	aboveskip=2em,
	abovecaptionskip=1em,
	captionpos=b,
	showstringspaces=false 
}

% Java Script Style
\lstdefinestyle{travis}{
	language=bash,
	numbers=left,
	stepnumber=5,
	firstnumber=1,
	tabsize=2,
	basicstyle=\ttm,
	breaklines=true,
	alsoletter={:},
	keywords={language:, node_js:, cache:, install:, before_script:, script:},
	keywordstyle=\ttb\color{orange},
	emph={MyClass,__init__},          % Custom highlighting
	emphstyle=\ttb\color{deepred},    % Custom highlighting style
	stringstyle=\color{deepgreen},
	frame=single,                         % Any extra options here
	backgroundcolor=\color{superlightgray},
	ndkeywords={directories:},
	ndkeywordstyle=\ttb\color{deepblue},
	belowskip=2em,
	aboveskip=2em,
	abovecaptionskip=1em,
	captionpos=b,
	showstringspaces=false 
}

% Java Script Style
\lstdefinestyle{cypress}{
	language=Java,
	numbers=left,
	stepnumber=5,
	firstnumber=1,
	tabsize=2,
	basicstyle=\ttm,
	breaklines=true,
	keywords={describe, before, beforeEach, context, it, expect, equal, cy, def},
	keywordstyle=\ttb\color{orange},
	emph={MyClass,__init__},          % Custom highlighting
	emphstyle=\ttb\color{deepred},    % Custom highlighting style
	stringstyle=\color{deepgreen},
	frame=single,                         % Any extra options here
	backgroundcolor=\color{superlightgray},
	ndkeywords={assert, break, case, catch, continue, debugger, default, delete, do, else, false, finally, for, function, if, in, instanceof, new, null, return, switch, this, throw, true, try, typeof, var, void, while, with},
	ndkeywordstyle=\ttb\color{deepblue},
	belowskip=2em,
	aboveskip=2em,
	abovecaptionskip=1em,
	captionpos=b,
	showstringspaces=false 
}

% Python Style
\lstdefinestyle{python}{
	language=python,
	numbers=left,
	basicstyle=\ttm,
	otherkeywords={self},
	keywordstyle=\ttb\color{deepblue},
	emph={MyClass,__init__},          % Custom highlighting
	emphstyle=\ttb\color{deepred},    % Custom highlighting style
	stringstyle=\color{deepgreen},
	frame=tb,                         % Any extra options here
	belowskip=3em,
	aboveskip=2em,
	abovecaptionskip=1em,
	captionpos=b,
	showstringspaces=false  
}

% Bash Style
\lstdefinestyle{bash}{
	language=bash,
	basicstyle=\ttm,
	numbers=left,
	stepnumber=5,
	firstnumber=1,
	backgroundcolor=\color{superlightgray},
	otherkeywords={},
	keywordstyle=\ttb\color{deepblue},
	emph={\$},          % Custom highlighting
	emphstyle=\ttb\color{deepred},    % Custom highlighting style
	stringstyle=\color{deepgreen},
	frame=single,                         % Any extra options here
	showstringspaces=false,
	belowskip=2em,
	aboveskip=2em,
	abovecaptionskip=1em,
	captionpos=b
}

%################################################################## DEFINITONS
%##################################################################
%################################################################## 

\newtcbtheorem[number within=chapter]{Definition}{}{
	enhanced,
	attach boxed title to top left={
		xshift=-1mm,
		yshift=-5mm,
		yshifttext=-1mm
	},
	breakable,
	top=0.5em,
	bottom=-0.5em,
	colback=white,
	colframe=black,
	fonttitle=\bfseries,
	boxed title style={
		size=small,
		colback=black,
		colframe=black,
	} 
}{def}

\newenvironment{definition}[2]{ \begin{Definition}[adjusted title=#1]{}{#2} 
		\textbf{} }{\end{Definition}}

%-----------------------------------------------------------------------------
% START OF DOCUMENT														---------------------
%-----------------------------------------------------------------------------
\begin{document}

% Deckblatt & Einverst�ndnisserkl�rung
\includepdf[pages={1}]{03_PDFs/Deckblatt.pdf}
\includepdf[]{03_PDFs/Erstellungserklaerung.pdf}

% Seitenzahlformatierung (rechts)
\headstyleright

% R�mische Ziffern
\roemischenummerierung

% Setze Seitenzahl
\setcounter{page}{3}	

% Danksagung
% !TEX root = Danksagung.tex

\section*{\huge\textbf{Danksagung}}
\secline
\\\\
HIER DANKSAGUNG

\newpage


% Kurzfassung
% !TEX root = Kurzfassung.tex

\section*{\huge\textbf{Kurzfassung}}
\secline
\\\\
Viele Unternehmen sind derzeit im Umbruch in Richtung Agilit�t, obwohl diese oft die Bedeutung noch nicht genau verstehen. Agilit�t wird von einigen geliebt, von anderen nicht verstanden und vom Rest kategorisch abgelehnt. Eine Studie hat gezeigt, dass agiles Arbeiten nicht nur das Teamklima verbessert, sondern Mitarbeiter auch visions-, aufgabenorientierter und innovationsfreudiger arbeiten. Ziel dieser Arbeit ist es nun, herauszufinden, was genau Unternehmen davon abh�lt, von einer klassischen auf eine agile Arbeitsweise umzusteigen. Um spezifischere Ergebnisse zu erhalten, wurden diese in die drei Typen, Start-up, Mittelstand und Konzern aufgeteilt. Anschlie�end wurde ein Fragebogen erarbeitet, der das Ziel hat, Herausforderungen und Metadaten zu den jeweiligen Unternehmenstypen zu ermitteln. Neben den Ergebnissen der Frageb�gen wurden mithilfe von Erfahrungsberichten und Interviews weitere Daten erhoben. Um aus den Daten, relevante Informationen herauszufiltern, wurde die qualitative Inhaltsanalyse nach Mayring angewendet. Diese hat 45 Herausforderungen und Metadaten der jeweiligen Unternehmenstypen zutage gebracht. Alle Herausforderungen wurden anschlie�end aus der Sicht jedes Unternehmenstyps mithilfe der Metadaten bewertet. Die Bewertung hat gezeigt, dass die gr��ten Herausforderungen f�r Start-ups in der Abh�ngigkeit zu den Investoren liegen. Der Mittelstand im Vergleich ist h�ufig traditionell gepr�gt, was Herausforderungen in der Unternehmenskultur und der Verkn�pfung von IT-Systemen mit sich bringt. Dagegen liegen die Herausforderungen von Konzernen, verursacht durch die oft durchwachsene Unternehmensstruktur, oft in der Komplexit�t, der Dauer der Transformation und der damit einhergehenden hohen Kosten. Trotz der Herausforderungen muss jedes Unternehmen f�r sich entscheiden, ob es den langen Weg einer Transformation gehen m�chte. 

\newpage


% Inhaltsverzeichnis
\inhaltsverzeichnis

% Kopf und Fu�zeile
\headstyle

% Abbildungsverzeichnis
\abbildungsverzeichnis

% Tabellenverzeichnis
\tabellenverzeichnis

% Acronyme
\abkuerzungen

% Glossar
\glossar

% Arabische Ziffern
\arabischenummerierung


%-----------------------------------------------------------------------------
% HAUPTARBEIT																	HAUPTARBEIT
%-----------------------------------------------------------------------------
% !TEX root = Einleitung.tex

\chapter{Einleitung}

Jedes Projekt mit dem Ziel Software zu entwickeln, ist individuell. Aus diesem Grund gibt es in den seltensten F�llen wiederholbare Prozesse. Reproduzierbare Prozesse k�nnen durch klare Anweisungen mithilfe anschlie�ender Kontrolle befehligt oder gesteuert werden. Kreative Prozesse hingegen scheitern bei diesem Ansatz, da diese individuell, je nach Situation gesteuert werden m�ssen. \autocite[vgl.][]{agiles_projektmanagement} Um genau diese Art von Steuerung erreichen zu k�nnen, versuchen immer mehr Unternehmen einen agilen Ansatz. Diese Vorgehensweise hat das Ziel, eine kreative Probleml�sung im Team zu f�rdern, da sich Kreativit�t nicht einfach anordnen l�sst \autocite[vgl.][S.VII]{agiler_fuehren}.
\\
\begin{definition}{Definition Agile F�hrung (Quelle: \autocite{agiler_fuehren})}{def:agile_fuehrung}
	\\ 
	Agile F�hrung unterst�tzt Mitarbeiter dabei, schnell und kreativ auf wechselnde Bed�rfnisse von Kunden und M�rkten zu reagieren. Sie ist ein Mindset, eine Haltung. Sie nutzt eine offene Toolbox mit Coachingwerkzeugen, die die Zusammenarbeit verbessern, sowie Methoden zur Reduktion von Komplexit�t.
	\\
\end{definition}

Unternehmens- und F�hrungskulturen sind derzeit in vielen Unternehmen im Umbruch in Richtung Agilit�t. Bei all den Ver�nderungen in diese Richtung wissen viele Unternehmen noch nicht, was es genau bedeutet, als Unternehmen, Team oder F�hrungskraft agiler zu werden. Eine Studie (siehe \autocite[vgl.][VII]{agiler_fuehren}) mit dem Namen \enquote{Teamklima f�r Innovation} wollte herausfinden, ob sich das Teamklima in nicht-agilen Gruppen von den agilen unterscheidet. Das Ergebnis hat gezeigt, dass nicht nur das Teamklima innerhalb agil arbeitender Teams besser war, sondern diese auch visions-, aufgabenorientierter und innovationsfreudiger agieren. Zudem verbessern agile Elemente wie Visualisierung, Teamentscheidung, Retrospektive, iterative Planung und Stand-up-Meetings die Zusammenarbeit \autocite[vgl.][S.VII-VIII]{agiler_fuehren}.
\\\\ 
Das Wort \enquote{Agil} ist eine Art Reizwort, das von einigen geliebt, von anderen nicht verstanden und vom Rest kategorisch abgelehnt wird. So schrieb die amerikanische Zeitschrift \enquote{Forbes!} (siehe \autocite{managers_hate_agile}), dass Manager \enquote{agile} (Agilit�t) hassen. Die Begr�ndung der Zeitschrift f�r die Abwehrhaltung der Manager lag in einem m�glichen Machtverlust, da Agilit�t im Management gerne mit dem Abbau von F�hrung verwechselt wird. Dabei geht es gar nicht um den strikten Abbau von F�hrung, sondern eher um das f�rdern von flacheren Hierarchien. H�ufig wird die Agilit�t abgelehnt, ohne genau zu wissen, was sich dahinter verbirgt. Diejenigen, die nur eine grobe Idee davon haben, begr�nden ihre ablehnende Haltung mit dem Argument, \enquote{alle machen, was sie wollen} und bef�rchten, dass das konzernweite Chaos ausbricht \autocite[vgl.][S.1]{agiler_fuehren}.
\\\\
Hinsichtlich der Ergebnisse der eben erw�hnten Studie (siehe \autocite[vgl.][VII]{agiler_fuehren} - Teamklima f�r Innovation) stellt sich nun die Frage, ob es nicht genau diese Eigenschaften sind, die ein \enquote{modernes} Projektmanagement in Zeiten, in denen Individualit�t so eine gro�e Rolle spielt, dabei helfen, scheitern zu verhindern. Wenn Agilit�t das Mittel ist, das die Eigenschaften liefert, die ben�tigt werden, um kreative Prozesse zu f�rdern, wieso wenden nicht alle Unternehmen agile Methodiken an? Zus�tzlich h�ren sich viele Argumente der Gegner anfangs plausibel an. Ob und wie viel Gewicht diese Argumente besitzen, wird im Laufe dieser Arbeit zus�tzlich ermittelt. Im n�chsten Abschnitt wird genau erl�utert, welche Hypothesen hinsichtlich der Herausforderung der agilen Transformation aufgestellt werden k�nnen.

\section{Zielsetzung} \label{Kap:Zielsetzung}
Ziel dieser Arbeit ist es herauszufinden, was genau Unternehmen davon abh�lt, agile Methodiken anzuwenden. Wieso f�llt der Umstieg augenscheinlich kleineren Unternehmen leichter als gro�en Unternehmen? Was genau unterscheidet kleine Unternehmen von gro�en Konzernen in dieser Hinsicht? F�r diese Arbeit werden die folgenden Hypothesen gebildet und im sp�teren Verlauf �berpr�ft:

\begin{itemize}
	\item \textbf{Die Gr��e des Unternehmens spielt eine Rolle}\\
	Desto kleiner ein Unternehmen ist, desto leichter f�llt die agile Transformation.
	
	\item \textbf{Gr��erer Innovationsdruck bei Konzernen}\\
	Je gr��er ein Konzern ist, desto mehr Innovationsdruck hin zu agilen Prozessen ist vorhanden.
	
	\item \textbf{Kosten der agilen Transformation}\\
	Die Kosten der agilen Transformation in Unternehmen, �bersteigen den Wert des Nutzen.
	
	\item \textbf{Unternehmensweites Chaos dank anarchischer Ans�tze}\\
	Bei der Einf�hrung von agilen Prozessen bricht das unternehmensweite Chaos aus.
\end{itemize}

Die �berpr�fung der Hypothesen ist nur ein Teil dieser Arbeit. So werden anhand verschiedenster Berichte, Artikel und Interviews die Herausforderungen der Agilen Transformation ermittelt. Zus�tzlich werden zu diesen Herausforderungen m�gliche L�sungsvorschl�ge erarbeitet. Um das Verst�ndnis f�r das Wort \enquote{Agilit�t} zu festigen, wird im n�chsten Kapitel genauer darauf und auf andere Grundlagen wie etwa klassische Projektmanagementmethoden und die Agile Transformation eingegangen. 




% R�mische Ziffern
\roemischenummerierung

% Setze Seitenzahl
\setcounter{page}{13}

% Literatur
\literaturverzeichnis
	
% Anhang
% !TEX root = Anhang.tex

\anhangstart

\begin{figure}[H]
	\centering
	\includegraphics[width=1.0\textwidth]{06_Bilder/Anhang_testdetail_output.png}
	\setlength{\abovecaptionskip}{-1em}
	\caption[]{Anhang Testdetail Output}
	\label{img:anhang_testdetail_output}
\end{figure}

\begin{figure}[H]
	\centering
	\includegraphics[width=1.0\textwidth]{06_Bilder/Anhang_testdetail_failures.png}
	\setlength{\abovecaptionskip}{-1em}
	\caption[]{Anhang Testdetail Failures}
	\label{img:anhang_testdetail_failures}
\end{figure}

\begin{figure}[H]
	\centering
	\includegraphics[width=1.0\textwidth]{06_Bilder/Anhang_testdetail_videos.png}
	\setlength{\abovecaptionskip}{-1em}
	\caption[]{Anhang Testdetail Videos}
	\label{img:anhang_testdetail_videos}
\end{figure}

\begin{figure}[H]
	\centering
	\includegraphics[width=1.0\textwidth]{06_Bilder/Anhang_testdetail_screenshots.png}
	\setlength{\abovecaptionskip}{-1em}
	\caption[]{Anhang Testdetail Screenshots}
	\label{img:anhang_testdetail_screenshots}
\end{figure}

\newpage

	\lstset{style=cypress, caption={test\_player.js, l�uft derzeit (07.03.2018) aktiv auf dem Travis}, label={lst:anhang_test_player}}
\begin{lstlisting}
describe('Player Tests', function() {

/**
* Get all the pre-stuff before every Test
*/
beforeEach(function(){
	cy.fixture('kraken/locators.json').as('locator')
	cy.fixture('kraken/web.json').as('web')
	cy.fixture('kraken/settings.json').as('setting')

})

context('Kraken', function(){

	/**
	* Krakentest with Ads
	*/
	it('With Ads', function(){

		const loc = this.locator
		const time = 10000

		// Visit the Website
		cy.visit(this.web.krakenUrl)

		// find the iFrame
		cy.iframe(loc.playerIframe, "#player")
			.as('player')

		// play the Video
		cy.get("@player")
			.find(loc.playButton, {timeout: time})
			.click()

		// scroll to top
		cy.scrollTo('top')
		
		// is the Ad displaying?
		cy.get("@player").find(loc.adDisclaimer, {timeout: time}) 
		cy.wait(2000).screenshot('Preroll')

		 // is the Video resuming after ad?
		cy.get("@player").find(loc.pauseButton, {timeout: time})

		// Test if the Video is playing
		cy.get("@player").find(loc.videoTimePassed)
			.contains("00:05", {timeout: 10000})

		// Test if pause is working
		cy.get("@player").find(loc.pauseButton).click()
		cy.get("@player").find(loc.playButtonBig)
			.should('be.visible')
		cy.scrollTo('top')
		cy.wait(1000).screenshot('Paused')
		cy.get("@player").find(loc.pauseButton).click()
		cy.scrollTo('top')

		// Test if Midroll is playing
		cy.get("@player")
			.find(loc.adDisclaimer, {timeout: 50000})
		cy.wait(2000).screenshot('Midroll')
		cy.get("@player")
			.find(loc.pauseButton, {timeout: time})

		// Test if Preroll is playing
		cy.get("@player")
			.find(loc.adDisclaimer, {timeout: 50000})
		cy.wait(2000).screenshot('Postroll')
	})
	
	/**
	* Krakentest without Ads
	*/
	it('Without Ads', function(){

		const loc = this.locator
		const time = 10000

		// Visit the Website and deactivate Hornet
		cy.visit(this.web.krakenUrl)
		cy.de_activate_hornet()

		// Find the iFrame
		cy.iframe(loc.playerIframe, "#player").as('player')
		
		// Play the Video
		cy.get("@player").find(loc.playButton).click()
		
		// Scroll to Top
		cy.scrollTo('top')

		// Test if the Video is playing
		cy.get("@player").find(loc.videoTimePassed)
			.contains("00:05", {timeout: time})

		// Test if pause is working
		cy.get("@player").find(loc.pauseButton).click()
		cy.get("@player").find(loc.playButtonBig)
			.should('be.visible')
		cy.scrollTo('top') // scroll to top
	})

	/**
	* Krakentest without Ads and Muted / Unmuted
	*/
	it('Mute / Unmute', function(){

		const loc = this.locator
		const time = 5000

		// Visit the Website and deaktivate Hornet
		cy.visit(this.web.krakenUrl)
		cy.de_activate_hornet()

		// Find the iFrame
		cy.iframe(loc.playerIframe, "#player").as('player')
		
		// Play the Video
		cy.get("@player").find(loc.playButton).click()
		
		// Scroll to Top
		cy.scrollTo('top')

		// Test if the Mute Button works
		cy.get("@player").find(loc.volumeButton).click()
		cy.get("@player").find(loc.volumeBar)
			.should('have.attr', 'style', 'transform: scaleX(0);')
		cy.scrollTo('top') // scroll to top
		cy.wait(500).screenshot('Muted')

		// Test if the Unmute Button works
		cy.get("@player").find(loc.volumeButton).click()
		cy.get("@player").find(loc.volumeBar)
			.should('have.attr', 'style', 'transform: scaleX(1);')
		cy.scrollTo('top') // scroll to top
		cy.wait(500).screenshot('Unmuted')
	})
})
})
\end{lstlisting} 

\newpage

	\lstset{style=cypress, caption={fixtures/kraken/locators.json}, label={lst:locators}}
\begin{lstlisting}
{
"playerIframe": "iframe__src-container-Player-styles__3QK78",
"playButton": "button[data-qa=playbackButton]",
"pauseButton": "button[data-qa=playbackButton]",
"playListButton": "button[data-qa=playlistButton]",
"pauseButton": "button[data-qa=playbackButton]",
"fullscreenButton": "button[data-qa=fullscreenButton]",
"volumeButton": "button[data-qa=muteButton]",
"adDisclaimer": "#player > div > div:nth-of-type(2) 
	> div:nth-of-type(2) > div > div > div > div 
	> div:nth-of-type(5) > div > div:nth-of-type(1) 
	> div:nth-of-type(1) > div > div > strong",
"videoTimePassed": "#player > div > div:nth-of-type(2) 
	> div:nth-of-type(2) > div > div > div > div 
	> div:nth-of-type(5) > div > div:nth-of-type(2) 
	> div:nth-of-type(1) > div > span",
"playButtonBig": "#player > div > div:nth-of-type(2) 
	> div:nth-of-type(2) > div > div > div > div 
	> div:nth-of-type(4) > div:nth-of-type(2) > div 
	> div > button"
"volumeBar": "#player > div > div:nth-of-type(2) 
	> div:nth-of-type(2) > div > div > div > div 
	> div:nth-of-type(5) > div > div:nth-of-type(2) 
	> div:nth-of-type(3) > div:nth-of-type(1) > div 
	> div > div:nth-of-type(1)"
}
\end{lstlisting} 

	\lstset{style=cypress, caption={fixtures/kraken/settings.json}, label={lst:settings}}
\begin{lstlisting}
{
"delay": 20000
}
\end{lstlisting} 

	\lstset{style=cypress, caption={fixtures/kraken/web.json}, label={lst:web}}
\begin{lstlisting}
{
"krakenUrl": "LINK ZUR KRAKEN TESTPAGE"
}
\end{lstlisting} 

	\lstset{style=cypress, caption={support/commands.js}, label={lst:commands}}
\begin{lstlisting}
Cypress.Commands.add('de_activate_hornet', () => {

cy.get('#hornet-input').click()
cy.scrollTo('top')
cy.reload()

})

Cypress.Commands.add('iframe', 
(locator, element, wait=10000) => {

return cy.get('iframe',  { timeout: wait })
.should('have.class', locator)
.should(($iframe) => {
expect($iframe.contents().find(element)).to.exist
}).then(($iframe) => {
return cy.wrap($iframe.contents().find("body"))
});

})
\end{lstlisting} 


%-----------------------------------------------------------------------------
% END OF DOCUMENT														  ---------------------
%-----------------------------------------------------------------------------
\end{document}

