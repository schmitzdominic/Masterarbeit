% !TEX root = Glossar.tex
\makeglossaries

\newglossaryentry{assertion}{ 
	name=Assertion, 
	description={ 
		Eine Assertion (englisch f�r Behauptung) markiert ein Ereignis im Quellcode, das eine Aussage �ber einen Zustand �berpr�ft. Entspricht der Wert dem vorgegebenen Zustand, wird dieser als erf�llt dargestellt oder zur�ckgegeben. Wird der vorgegebene Zustand nicht erf�llt, wird entsprechend ein Fehler zur�ckgegeben \autocite{assertion}} 
} 

\newglossaryentry{bash}{ 
	name=Bash, 
	description={ 
		Eine Bash ist eine frei erh�ltliche Unix-Shell (Eingabebasierte Benutzerschnittstelle). Der Name setzt sich aus den Worten Bourne again Shell zusammen und ist heutzutage auf vielen Linux Systemen die Standard-Shell \autocite{bash}}
}  

\newglossaryentry{branch}{ 
	name=Branch, 
	description={
		Das Versionsverwaltungstool Git beispielsweise, erstellt zur Versionierung eine Serie von Commits. Um parallel an mehreren Codest�nden arbeiten zu k�nnen, k�nnen bewegbare Zeiger (Branches) erstellt werden, die auf verschiedene \glspl{commit} zeigen \autocite{branch}} 
} 

\newglossaryentry{browser}{ 
	name=Browser, 
	description={ 
		Ein Browser auch Webbrowser genannt, stellt Websiten durch das �bersetzen von beispielsweise HTML Dateien dar \autocite{browser}} 
} 

\newglossaryentry{commit}{ 
	name=Commit, 
	description={ 
		Innerhalb eines Versionsverwaltungstools wie beispielsweise Git, repr�sentiert ein commit, einen Status (Schnappschuss) eines Softwareprojekts. W�hrend der Ausf�hrung eines commit Befehls, wird der Zustand s�mlticher Dateien eines Softwareprojekts gesichert \autocite{gitbook}} 
} 

\newglossaryentry{crossBrowser}{ 
	name=Cross-Browser-Testing, 
	description={ 
		Beim Cross-Browser-Testing wird der Inhalt einer Webseite, auf mehreren Browsern getestet, um zu gew�hrleisten, dass dieser sich unabh�ngig vom Browsertyp identisch verh�lt \autocite{crossBrowser}} 
} 

\newglossaryentry{fixture}{ 
	name=Fixture, 
	description={
		Eine Fixture initialisiert bestimmte Werte oder bereitet ein System f�r einen Test vor. Fixtures k�nnen au�erdem Werte oder Objekte die eine Relevanz f�r einen Test besitzen, zur�ck geben \autocite{fixture}} 
} 

\newglossaryentry{iframe}{ 
	name=iFrame, 
	description={ 
		Ein iFrame auch InlineFrame genannt, dient der Strukturierung von Webseiten und stellt weitere Webinhalte als eigenst�ndige Dokumente innerhalb eines definierten Bereichs eines Browsers dar \autocite{iFrame}} 
} 

\newglossaryentry{locator}{ 
	name=Locator, 
	description={ 
		Eindeutige Beschreibung eines Websiten Elements wie beispielsweise ein Button, eine �berschrift oder ein Listenelement \autocite{locator}} 
}

\newglossaryentry{packagemanager}{ 
	name=Paketmanager, 
	description={ 
		Mithilfe eines Paketmanagers auch oft Paketverwaltung genannt, ist es m�glich Softwarepakete die zum Download bereitgestellt werden, komfortabel zu verwalten (Installieren, Aktualisieren, Deinstallieren) \autocite{paketmanager}} 
} 

\newglossaryentry{parser}{ 
	name=Parser, 
	description={ 
		Ein Parser ist ein Teil eines Compilers und ist f�r die Aufsplittung und Umwandlung eines Codes zust�ndig. Dadurch ist es m�glich beispielsweise XML Dokumente entsprechend aufzubereiten \autocite{parser}} 
}   

\newglossaryentry{plugin}{ 
	name=Plugin, 
	description={ 
		Eine optionale Komponente um bestehende Software zu erweitern oder zu ver�ndern, wird Plugin gennant. Ein Merkmal ist die Abh�ngigkeit der Hauptanwendung \autocite{plugin}} 
}  

\newglossaryentry{repository}{ 
	name=Repository, 
	description={ 
		Ein Repository bezeichnet eine zentrale Ablage f�r Dokumente oder Daten. Oft werden sie zum Dokumentenmanagement, Content-Management oder in der Versionsverwaltung eingesetzt \autocite{repository}} 
}  

\newglossaryentry{snapshot}{ 
	name=Snapshot, 
	description={ 
		Ein Snapshot auch Schnappschuss genannt, beschreibt eine Momentaufnahme eines Objekts oder Systems \autocite{snapshot}} 
} 

\newglossaryentry{syntax}{ 
	name=Syntax-Highlighting, 
	description={ 
		Beim Syntax-Highlighting werden bestimmte W�rter und Zeichenkombinationen innerhalb eines Textes, abh�ngig ihrer Bedeutung, in unterschiedlichen Farben hervorgehoben \autocite{syntax}} 
} 

\newglossaryentry{thread}{ 
	name=Thread, 
	description={ 
		Ein Thread bezogen auf Internetforen, bezeichnet ein Thema. H�ufig wird ein Thead mit einem Thema oder einer Frage er�ffnet und andere Benutzer k�nnen antworten \autocite{thread}} 
} 

\newglossaryentry{workaround}{ 
	name=Workaround, 
	description={ 
		Ein Workaround (zu deutsch Problemumgehung) ist eine vorl�ufige L�sung eines bekannten technischen Problems. Das Problem wird hierbei nicht behoben, sondern nur ein Weg gefunden, es mit zus�tzlichen Aufwand zu umgehen \autocite{workaround}} 
} 
