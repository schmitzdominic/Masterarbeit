% !TEX root = Glossar.tex
\makeglossaries

\newglossaryentry{venturecapital}{ 
	name=Venture-Capital, 
	description={ 
		Unter Venture Capital versteht man die Finanzierung eines nicht b�rsennotierten Unternehmens mit Eigenkapital. Dabei wird das Unternehmen in der Regel zum Zeitpunkt der Beteiligung privat gehalten. Entscheidend ist, dass das Eigentum am Unternehmen w�hrend der meisten Zeit der Beteiligung in privaten H�nden liegt \autocite{venture_capital}}
}

\newglossaryentry{servantleader}{ 
	name=Servant Leader, 
	description={ 
		Ein Servant Leader liebt Menschen und m�chte ihnen helfen. Die Aufgabe eines Servant Leaders ist es daher, die Bed�rfnisse anderer zu ermitteln und zu versuchen, diese Bed�rfnisse zu befriedigen \autocite{servant_leadership}}
} 

\newglossaryentry{iteration}{ 
	name=Iteration, 
	description={ 
		Eine Iteration ist eine wiederholte Durchf�hrung eines Vorgangs. Die Anzahl der Durchf�hrungen (Iterationen) steht entweder vorher fest oder richtet sich nach der Erf�llung eines Abbruchkriteriums \autocite{definition_iteration}}
} 

\newglossaryentry{stakeholder}{ 
	name=Stakeholder, 
	description={ 
		Anspruchsgruppen sind alle internen und externen Personengruppen, die von den unternehmerischen T�tigkeiten gegenw�rtig oder in Zukunft direkt oder indirekt betroffen sind. Gem�� Stakeholder-Ansatz wird ihnen - zus�tzlich zu den Eigent�mern (Shareholders) - das Recht zugesprochen, ihre Interessen gegen�ber der Unternehmung geltend zu machen \autocite{definition_stakeholder}}
}

\newglossaryentry{trialanderror}{ 
	name=Trial and Error, 
	description={ 
		Trial and Error beschreibt einen Weg, ein Ziel zu erreichen oder ein Problem zu l�sen, indem verschiedene Methoden ausprobiert und aus Fehlern gelernt wird \autocite{definition_trialanderror}}
}

\newglossaryentry{governance}{ 
	name=Governance, 
	description={ 
		Die Art und Weise, wie in einem Land die Angelegenheiten der Allgemeinheit verwaltet und geregelt werden. \enquote{Governance} ist nicht \enquote{Government} gleich \enquote{Regierung}, sondern umfassender und offener \autocite{definition_governance}}
}

\newglossaryentry{workaround}{ 
	name=Workaround, 
	description={ 
		Ein workaround beschreibt eine M�glichkeit, ein Problem zu l�sen, oder etwas trotz des Problems funktionsf�hig zu bekommen, ohne es vollst�ndig zu l�sen \autocite{definition_workaround}}
}

\newglossaryentry{coredeliverable}{ 
	name=Core Deliverable, 
	description={ 
		Ein Core Deliverable ist das zentrale Produkt oder Dienstleistung, welches einem Kunden zu Verf�gung gestellt wird \autocite{definition_deliverable}}
}

\newglossaryentry{cicd}{ 
	name=Continous Integration, 
	description={ 
		Continous Integration beschreibt einen Softwareentwicklungsprozess, bei dem Teammitglieder h�ufig ihre Arbeit integrieren. Bei jeder Integration wird der Stand automatisch �berpr�ft, um Integrationsfehler so schnell wie m�glich zu erkennen \autocite{definition_ci}}
}

\newglossaryentry{devops}{ 
	name=DevOps, 
	description={ 
		Der Begriff besteht aus Development (Dev), der den Softwareentwickler beschreibt, und Operations (Ops), was den IT-Betrieb darstellt. Aus der Kombination beider soll ein Prozessverbesserungsansatz f�r die Bereiche Softwareentwicklung und Systemadministration entstehen. Es wird eine effektivere und effizientere Zusammenarbeit der verschiedenen Bereiche angestrebt \autocite{definition_devops}}
}
