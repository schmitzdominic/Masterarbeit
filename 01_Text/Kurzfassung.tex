% !TEX root = Kurzfassung.tex

\section*{\huge\textbf{Kurzfassung}}
\secline
\\\\
Viele Unternehmen sind derzeit im Umbruch in Richtung Agilit�t, obwohl diese oft die Bedeutung noch nicht genau verstehen. Agilit�t wird von einigen geliebt, von anderen nicht verstanden und vom Rest kategorisch abgelehnt. Eine Studie hat gezeigt, dass agiles Arbeiten nicht nur das Teamklima verbessert, sondern Mitarbeiter auch visions-, aufgabenorientierter und innovationsfreudiger arbeiten. Ziel dieser Arbeit ist es nun, herauszufinden, was genau Unternehmen davon abh�lt, von einer klassischen auf eine agile Arbeitsweise umzusteigen. Um spezifischere Ergebnisse zu erhalten, wurden diese in die drei Typen, Start-up, Mittelstand und Konzern aufgeteilt. Anschlie�end wurde ein Fragebogen erarbeitet, der das Ziel hat, Herausforderungen und Metadaten zu den jeweiligen Unternehmenstypen zu ermitteln. Neben den Ergebnissen der Frageb�gen wurden mithilfe von Erfahrungsberichten und Interviews weitere Daten erhoben. Um aus den Daten, relevante Informationen herauszufiltern, wurde die qualitative Inhaltsanalyse nach Mayring angewendet. Diese hat 45 Herausforderungen und Metadaten der jeweiligen Unternehmenstypen zutage gebracht. Alle Herausforderungen wurden anschlie�end aus der Sicht jedes Unternehmenstyps mithilfe der Metadaten bewertet. Die Bewertung hat gezeigt, dass die gr��ten Herausforderungen f�r Start-ups in der Abh�ngigkeit zu den Investoren liegen. Der Mittelstand im Vergleich ist h�ufig traditionell gepr�gt, was Herausforderungen in der Unternehmenskultur und der Verkn�pfung von IT-Systemen mit sich bringt. Dagegen liegen die Herausforderungen von Konzernen, verursacht durch die oft durchwachsene Unternehmensstruktur, oft in der Komplexit�t, der Dauer der Transformation und der damit einhergehenden hohen Kosten. Trotz der Herausforderungen muss jedes Unternehmen f�r sich entscheiden, ob es den langen Weg einer Transformation gehen m�chte. 

\newpage
