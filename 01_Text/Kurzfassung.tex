% !TEX root = Kurzfassung.tex

\section*{\huge\textbf{Kurzfassung}}
\secline
\newline\newline
Die meisten Menschen, die sich schon einmal mit der Testautomatisierung von Weboberfl�chen besch�ftigt haben, haben bereits von Selenium geh�rt. Seit dem Jahre 2015 scheint es eine Alternative namens Cypress zu geben. Das Unternehmen glomex GmbH versucht herauszufinden, ob diese Alternative, das momentan eingesetzte System abl�sen kann. Momentan wird ein Verbund aus Selenium, Pytest und Python zur Testautomatisierung eingesetzt. Um beide Systeme miteinander zu vergleichen, wurden die f�r das Unternehmen wichtigsten Kriterien Installation, Abh�ngigkeiten, Ausf�hrung w�hrend der Testerstellung, Kompatibilit�t und \gls{iframe} Support aufgestellt. Jedes System wurde auf die genannten Kriterien hin gepr�ft und mit einer Punktzahl zwischen 1 (ungeeignet) bis 3 (gut geeignet) versehen. Abschlie�end wurde eine Gesamtpunktzahl je System ermittelt, um herauszufinden, welches sich besser eignet. Dabei hat sich herausgestellt, dass sich Selenium zum momentanen Zeitpunkt (Stand 07.03.2018) besser f�r den Einsatz bei der glomex GmbH eignet. W�hrend dem Vergleich wurde deutlich, dass Cypress besonders bei den zwei wichtigsten Kriterien, Kompatibilit�t und dem \gls{iframe} Support nachholen muss. Cypress arbeitet bereits an diesen Punkten und k�nnte so zuk�nftig zu einem sehr guten Alternativsystem werden.

\newpage
