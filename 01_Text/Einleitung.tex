% !TEX root = Einleitung.tex

\chapter{Einleitung}

Jedes Projekt mit dem Ziel Software zu entwickeln, ist individuell. Aus diesem Grund gibt es in den seltensten F�llen wiederholbare Prozesse. Reproduzierbare Prozesse k�nnen durch klare Anweisungen mithilfe anschlie�ender Kontrolle befehligt oder gesteuert werden. Kreative Prozesse hingegen scheitern bei diesem Ansatz, da diese individuell, je nach Situation gesteuert werden m�ssen. \autocite{agiles_projektmanagement}

\section{Motivation} \label{kap:Motivation}
Leser Motivieren!

\section{Zielsetzung} \label{Kap:Zielsetzung}
Was f�r Ziele m�chte ich in dieser Arbeit erreichen?



\chapter{Grundlagen}

\section{Agilit\"at} \label{Kap:Agilitaet}
agilit�t

\section{Agile Projektmanagementmethoden} \label{Kap:AgileProjektmanagementmethoden}
Methoden
\subsection{SCRUM} \label{Kap:Scrum}
scrum

\subsection{Kanban} \label{Kap:Kanban}
Kanban

\section{Agilit\"at in gro�en Unternehmen} \label{Kap:AgilitaetGrosseUnternehmen}
Was f�r Formen werden in gro�en Teams verwendet? SCRUM in SCRUM? usw.. Recherche!

\section{Klassische Projektmanagementmethoden} \label{Kap:KlassischeProjektmanagementmethoden}
Klassische Projektmanagementmethoden
V Modell?
Wasserfall?
G�ngigste Methoden herausfinden und definieren. (Evtl. darauf achten welche Methoden die Unternehmen sp�ter in den Berichten und Interviews verwenden und diese hier erl�utern)

\section{Agile Transformation} \label{Kap:AgileTransformation}
Was ist die Agile Transformation? Bitte IT Kontext beibehalten!



\chapter{Einteilung der Unternehmen}

\section{Startup} \label{Kap:Startup}
Was ist ein Startup? Nach welchen Definitionen darf ein Unternehmen ein Startup genannt werden?
Pr�gnante Merkmale?
Wie viele Mitarbeiter?
Umsatz?

\subsection{G\"angige Projektmanagementmethoden in Startups} \label{Kap:MethodenStartups}
Methoden

\section{Mittelstand} \label{Kap:Mittelstand}
Was ist der Mittelstand? Nach welchen Definitionen darf ein Unternehmen sich Mittelstand nennen?
Pr�gnante Merkmale?
Wie viele Mitarbeiter?
Umsatz?

\subsection{G\"angige Projektmanagementmethoden im Mittelstand} \label{Kap:MethodenMittelstand}
Methoden

\section{Konzern} \label{Kap:Konzern}
Wann ist ein Unternehmen ein Konzern? Was unterscheidet einen Konzern von den anderen Typen?
Pr�gnante Merkmale?
Wie viele Mitarbeiter?
Umsatz?

\subsection{G\"angige Projektmanagementmethoden in Konzernen} \label{Kap:MethodenKonzerne}
Methoden


\chapter{Erfahrungsberichte der Agilen Transformation}

\section{Berichte} \label{Kap:Berichte}
Berichte �ber bereits durchgef�hrte Agile Transformationen

\section{Interviews} \label{Kap:Interviews}
Unternehmen, Menschen oder Internet evtl. auch B�cher

\section{Merkmale und Schwierigkeiten} \label{Kap:MerkmaleUndSchwierigkeiten}
Welche Merkmale und Schwierigkeiten haben sich w�hrend der Berichte und Interviews herauskristallisiert?

\chapter{Schwierigkeiten der Agilen Transformation}

\section{Klassifizierung und Sortierung} \label{Kap:Klassifizierung}
Jede Schwierigkeit klassifizieren und am ende nach Relevanz sortieren (Was sind die Hauptgr�nde das eine Agile Transformation nicht klappt?)

\section{Zuordnung nach Unternehmenstyp} \label{Kap:ZuordnungSchwierigkeiten}
Welche Schwierigkeiten haben welche Unternehmenstypen?
Korrelationen?

\section{Ausl\"oser der Schwierigkeiten} \label{Kap:AusloeserSchwierigkeiten}
Wieso treten diese Schwierigkeiten auf?
Treten diese Schwierigkeiten nur bei bestimmten Unternehmenstypen auf?

\section{Ratschl\"age zur Beseitigung der Schwierigkeiten} \label{Kap:Ratschlaege}
Wie k�nnen die Schwierigkeiten gemindert oder gar komplett vermieden werden?

\chapter{Zukunft der Agilen Transformation} \label{Kap:ZukunftAgileTransformation}
Wie l�uft die Zukunft ab? 
Ist die Agile Transformation nur ein Trend?
Wie ver�ndert sich die Agile Transformation in der IT in den n�chsten Jahren?

\chapter{Fazit}

\section{Reflexion} \label{Kap:Reflexion}
Was sind die Ergebnisse dieser Arbeit?

\section{Ausblick} \label{Kap:Ausblick}
Ausblick im Bezug auf die Ergebnisse!

\newpage


