% !TEX root = Einleitung.tex

\chapter{Einleitung}

\section{Motivation}
Je umfangreicher und komplexer Software-Systeme sind, desto fehleranf�lliger werden diese. Jedes softwareproduzierende Unternehmen bem�ht sich daher, eine entsprechend hohe Qualit�t zu gew�hrleisten, damit der Kunde letztendlich kein fehlerhaftes Produkt erh�lt. Aus diesem Grund ist es essenziell, Software auf ihre Funktionalit�t hin zu testen. 
\newline\newline
Die vielen Abh�ngigkeiten innerhalb moderner Software-Systeme machen es notwendig nach jeder �nderung am Quellcode, die betroffenen Bereiche erneut zu testen. Das wiederum f�hrt zu einem enormen Aufwand, da h�ufig das komplette Produkt wiederholt getestet werden muss \autocite{softwarequalitaet}. Durch die Automatisierung von Tests versprechen sich viele Unternehmen eine Zeitersparnis des Aufwandes. Daher wird meistens eine Kombination aus bestimmten Frameworks verwendet, die in Verbindung ganze Webanwendungen testen k�nnen und in der Lage sind, alle Funktionalit�ten und Abh�ngigkeiten einer Software zu testen.

\section{Zielsetzung}
Ziel dieser Arbeit ist die Evaluation eines neuen Frameworks namens \enquote{Cypress} (auch Cypress.io genannt), um den Testaufwand bei der glomex GmbH zu verringern. Mithilfe von Cypress ist es m�glich Oberfl�chen von Webanwendungen (graphisch) im Browser zu testen. Bisher wurde hierf�r die Programmiersprache Python in Verbindung mit Selenium und Pytest verwendet. Da alle Produkte der glomex beim Kunden zum Einsatz kommen, sind w�hrend der Produktentwicklung andere Browserplattformen zu behandeln als im Unternehmen selbst. Es sollen alle wichtigen Unterschiede und eventuelle positive sowie negative Aspekte von Cypress gegen�ber dem momentanen System herausgearbeitet und aufgezeigt werden.  Des Weiteren wird ermittelt, ob und wie sich die neuen Tests automatisiert ausf�hren lassen und inwiefern Cypress L�sungen in diesem Bereich bereitstellt. 

\newpage

\section{Das Unternehmen glomex GmbH}
Das Unternehmen glomex GmbH, im Weiteren glomex genannt, ist eine Tochter der ProsiebenSat1. Media Se mit Hauptsitz in der Landsberger Str. 110 in M�nchen. Die glomex ist ein technischer Dienstleister zur Onlineverbreitung von Premium Videos. Das Unternehmen erm�glicht es Anbietern und Verbreitern von Videos, Inhalte zu syndizieren. In Form einer cloudbasierten Transaktionsplattform wird dazu die technische Infrastruktur zur Verf�gung gestellt. Auf dieser Plattform, ist es m�glich Videos zur Verf�gung zu stellen, die wiederum von Verbreitern (Publishern) auf ihren Websites oder Apps eingebunden werden k�nnen. Um dies zu erm�glichen, bietet die glomex au�erdem eine Video-Delivery-Infrastruktur zur Verbreitung von Videos, mit deren Hilfe der weltweite Content distribuiert und monetarisiert wird. Durch die signifikante Steigerung der Reichweite werden neue Ums�tze erschlossen und gleichzeitig die Betriebskosten f�r Videoauftritte und Verbreitung reduziert \autocite{glomexAbout}.\\\\ 

\begin{figure}[H]
	\centering
	\includegraphics[width=1.0\textwidth]{06_Bilder/glomex.png}
	\setlength{\abovecaptionskip}{-1em}
	\caption{glomex Webseite (Quelle: \autocite{glomex})}
	\label{img:glomex}
\end{figure} 

