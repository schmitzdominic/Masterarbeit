% !TEX root = Old.tex


% HIER ARCHIVIERTEN TEXT REIN
\subsection{G\"angige Projektmanagementmethoden in Startups ÜBERLEGEN OB ÜBERHAUPT NOTWENDIG!} 

\subsection{G\"angige Projektmanagementmethoden im Mittelstand ÜBERLEGEN OB ÜBERHAUPT NOTWENDIG!} \label{Kap:MethodenMittelstand}

\subsection{G\"angige Projektmanagementmethoden in Konzernen} \label{Kap:MethodenKonzerne}
Methoden


	\begin{itemize}
	\item \textbf{Eliminate Waste}\\
	Aufgaben werden von Personen genommen. Das ermöglicht es dem Bearbeiter, erst dann Aufgaben zu nehmen, wenn dieser bereit dafür ist. Es wird vermieden an mehreren Aufgaben zur gleichen Zeit zu arbeiten \autocite[vgl.][S.55]{kanban}. 
	
	\item \textbf{Amplify Learning}\\
	Alle Personen bestimmten selber, wann diese mit der Bearbeitung beginnen. Eine Überlastung wird dadurch vermieden. Zusätzlich kann dieser Freiraum genutzt werden um zu Lernen, sofern dies Vorrang vor der Bearbeitung der anstehenden Aufgabe hat \autocite[vgl.][S.55]{kanban}. 
	
	\item \textbf{Decide as Late as Possible}\\
	Eine Aufgabe wird erst dann zur Bearbeitung in die nächste Phase gezogen, wenn diese auch direkt bearbeitet werden kann. Aufgaben verweilen solange es nötig ist in allen Phasen. Während der Zeit in der Beraten wird, welche Aufgaben als nächstes bearbeitet werden, können sich weitere Informationen zu den Aufgaben ergeben (passiv) oder (aktiv) eingeholt werden. Die Entscheidung welche Aufgaben gezogen werden, kann durch diese Informationen erleichtert werden \autocite[vgl.][S.55,56]{kanban}.
\end{itemize}  