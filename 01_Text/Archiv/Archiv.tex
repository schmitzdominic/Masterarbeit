% !TEX root = Old.tex


% HIER ARCHIVIERTEN TEXT REIN
\subsection{G\"angige Projektmanagementmethoden in Startups ÜBERLEGEN OB ÜBERHAUPT NOTWENDIG!} 

\subsection{G\"angige Projektmanagementmethoden im Mittelstand ÜBERLEGEN OB ÜBERHAUPT NOTWENDIG!} \label{Kap:MethodenMittelstand}

\subsection{G\"angige Projektmanagementmethoden in Konzernen} \label{Kap:MethodenKonzerne}
Methoden


	\begin{itemize}
	\item \textbf{Eliminate Waste}\\
	Aufgaben werden von Personen genommen. Das ermöglicht es dem Bearbeiter, erst dann Aufgaben zu nehmen, wenn dieser bereit dafür ist. Es wird vermieden an mehreren Aufgaben zur gleichen Zeit zu arbeiten \autocite[vgl.][S.55]{kanban}. 
	
	\item \textbf{Amplify Learning}\\
	Alle Personen bestimmten selber, wann diese mit der Bearbeitung beginnen. Eine Überlastung wird dadurch vermieden. Zusätzlich kann dieser Freiraum genutzt werden um zu Lernen, sofern dies Vorrang vor der Bearbeitung der anstehenden Aufgabe hat \autocite[vgl.][S.55]{kanban}. 
	
	\item \textbf{Decide as Late as Possible}\\
	Eine Aufgabe wird erst dann zur Bearbeitung in die nächste Phase gezogen, wenn diese auch direkt bearbeitet werden kann. Aufgaben verweilen solange es nötig ist in allen Phasen. Während der Zeit in der Beraten wird, welche Aufgaben als nächstes bearbeitet werden, können sich weitere Informationen zu den Aufgaben ergeben (passiv) oder (aktiv) eingeholt werden. Die Entscheidung welche Aufgaben gezogen werden, kann durch diese Informationen erleichtert werden \autocite[vgl.][S.55,56]{kanban}.
\end{itemize}  


\section{Kategorisierung} \label{Kap:Kategorisierung}

Um die Herausforderungen übersichtlicher zu gestalten, werden diese in verschiedene Kategorien eingeteilt und sortiert. Die Kategorien werden mithilfe der Frage \enquote{Wer ist unmittelbar von der Herausforderung betroffen} gebildet. So ergeben sich die folgenden Kategorien:

\begin{itemize}
	\item Unternehmen (Betrifft das komplette Unternehmen)
	\item Management (Betrifft nur das Management)
	\item Team (Betrifft einzelne Teams)
	\item Mitarbeiter (Betrifft direkt den Mitarbeiter)
	\item Undefinierbar
\end{itemize}

\setlength\LTleft{-1in}
\setlength\LTright{-1in}
\begin{longtable}{|p{1,1cm}|p{5cm}|p{12,4cm}|}
	\caption{Kategorisierung der Herausforderungen} \label{tbl:kategorisierung} \\
	
	\hline \multicolumn{1}{|c|}{\textbf{Hf.}} 
	& \multicolumn{1}{c|}{\textbf{Kategorie}}
	& \multicolumn{1}{c|}{\textbf{Kurzbegründung}} \\ \hline 
	\endfirsthead
	
	\multicolumn{3}{c}%
	{{\bfseries \tablename\ \thetable{} -- weiterführend von letzter Seite}} \\
	\hline \multicolumn{1}{|c|}{\textbf{Hf.}} 
	& \multicolumn{1}{c|}{\textbf{Kategorie}}
	& \multicolumn{1}{c|}{\textbf{Kurzbegründung}} \\ \hline 
	\endhead
	
	\hline \multicolumn{3}{|r|}{{Wird auf der nächsten Seite weitergeführt}} \\ \hline
	\endfoot
	
	\hline \hline
	\endlastfoot
	
	K1.1 & Unternehmen
	& Die Begründung, warum eine agile Transformierung stattfindet, betrifft das komplette Unternehmen.
	\\ \hline
	K1.2 & Unternehmen
	& Das komplette Unternehmen muss lernen, richtig einzuschätzen, ob agile oder klassische Ziele während der Transformation Vorrang haben.
	\\ \hline
	K1.3 & Unternehmen
	& Wenn das Tagesgeschäft nicht mehr weitergeführt wird, ist das komplette Unternehmen betroffen.
	\\ \hline
	K1.7 & Management
	& Das Management organisiert im klassischen und muss erst lernen selbstorganisierende Mitarbeiter zu tolerieren. 
	\\ \hline
	K1.4 & Team
	& Die Transformation bringt völlig neue Verantwortung für das Team.
	\\ \hline
	K1.5 & Team
	& Die Transformation erfordert Aufmerksamkeit innerhalb der Teams.
	\\ \hline
	K1.6 & Team
	& Die Transformation bringt völlig neue Verantwortung für das Team.
	\\ \hline
	8. & Mitarbeiter müssen hinreichend in Richtung Agilität geschult werden.
	& 
	\\ \hline
	9. & Unternehmen müssen bereit sein, alte Strukturen aufzubrechen und neue Wege zu gehen (Prozesse).
	&
	\\ \hline
	10. & Unternehmen müssen lernen richtig mit Feedback umzugehen.
	&
	\\ \hline
	11. & Im Zuge der agilen Transformation muss das Unternehmen in der Lage sein, agile Projekte richtig in klassische Kulturen zu integrieren.
	&
	\\ \hline
	12. & Unternehmen müssen abschätzen ob die eigene Organisation überhaupt für eine agile Transformation geeignet ist.
	&
	\\ \hline
	13. & Mitarbeiter haben häufig ein falsches Verständnis von Agilität und ihren Werte, Prinzipien und Methoden.
	&
	\\ \hline
	14. & Teams haben plötzlich die alleinige Verantwortung für ihr Inkrement.
	&
	\\ \hline
	15. & Unternehmen müssen ihre Kultur und Umgebung in Richtung Agilität anpassen (Kultur und Umgebung).
	&
	\\ \hline
	16. & Unternehmen müssen in der Lage sein, bereits bestehende Probleme in der Unternehmenskultur zu lösen.
	&
	\\ \hline
	17. & Teams haben plötzlich keine Projektpläne mehr, an die sie sich halten müssen, da im agilen keine Projektpläne existieren.
	&
	\\ \hline
	18. & Mitarbeitern erscheint die Selbstorganisation anfangs unmöglich.
	&
	\\ \hline
	19. & Unternehmen müssen lernen, die interne Organisation zu fixieren, trotzt stetig wechselnder Projektteams.
	&
	\\ \hline
	20. & Unternehmen müssen sich stetig neu ausrichten.
	&
	\\ \hline
	21. & Unternehmen müssen trotz des Wandels darauf achten, nicht jede technische Neuerung mitzunehmen.
	&
	\\ \hline
	22. & Unternehmen dürfen nie den Fokus auf das wesentliche verlieren.
	&
	\\ \hline
	23. & Unternehmen müssen dauerhaft Trends beobachten.
	&
	\\ \hline
	24. & Tochter- und Mutterkonzerne besitzen häufig Schnittpunkte die jeweils klassisch oder agil sein können. Unternehmen müssen eine Lösung für diese unterschiedlichen Schnittpunkte finden, um die agile Transformation nicht zu gefährden.
	&
	\\ \hline
	25. & Das Topmanagement hat Angst vor einem Machtverlust, durch die flacher werdenden Hierarchien, verursacht durch die agile Transformation.
	&
	\\ \hline
	26. & Die Agilen Werte müssen international an alle Mitarbeiter vermittelt werden.
	&
	\\ \hline
	27. & Während der agilen Transformation kommt es kurzfristig zu Abstrichen in der Produktivität, verursacht durch das Investment in die Transformation.
	&
	\\ \hline
	28. & Unternehmen müssen lernen, externe Mitarbeiter einzugliedern.
	&
	\\ \hline
	29. & Die agile Transformation erfordert es, Gegner zu überzeugen.
	&
	\\ \hline
	30. & Mitarbeiter müssen aktiviert werden, selber Verantwortung zu übernehmen.
	&
	\\ \hline
	31. & Unternehmen müssen lernen Projekte nicht nur agil aufzusetzen, sondern diese auch agil umzusetzen.
	&
	\\ \hline
	32. & Continous delivery muss auch bei nicht agilen Prozessen stattfinden.
	&
	\\ \hline
	33. & Bereits bestehende langfristige klassische Projekte müssen auf Agile umgestellt werden.
	&
	\\ \hline
	34. & Das Mitarbeitermindset muss sich in Richtung Agile ändern.
	&
	\\ \hline
	35. & Agile Vorgehensweisen sind anstrengender als klassische, da dauerhaft neue Releases stattfinden.
	&
	\\ \hline
	36. & Die Vorteile einer guten Fehlerkultur müssen erkannt werden.
	&
	\\ \hline
	37. & Mitarbeiter müssen lernen, sich häufig abzustimmen, trotz eigener Projekte und dem Tagesgeschäft.
	&
	\\ \hline
	38. & Traditionelle Deutsche Unternehmen sehen Fehler noch nicht als Chance, sondern eher als Fehlschlag.
	&
	\\ \hline
	39. & Mitarbeitern muss die Angst vor Fehlern genommen werden.
	&
	\\ \hline
	40. & Der Gedanke \enquote{Bis jetzt hat es immer so geklappt, warum sollten wir daran etwas ändern} muss beseitigt werden.
	&
	\\ \hline
	41. & Es muss kein Zeitraum mehr festgelegt werden, da die Projekte agil sind.
	&
	\\ \hline
	42. & In vielen Unternehmen fehlt ein Vergleich zwischen agilen und nicht agilen Projekten.
	&
	\\ \hline
	43. & Die Geschäftsführung muss aktiv hinter der agilen Transformation stehen.
	&
	\\ \hline
	44. & Auch Mitarbeitern, die mehr Zeit für Aufgaben benötigen, muss eine entsprechende Toleranz entgegengebracht werden.
	&
	\\ \hline
	45. & Mitarbeiter haben weniger Zeit für das Tagesgeschäft.
	&
	\\ \hline
	46. & Die Vorteile der Agilität müssen klar ersichtlich sein.
	&
	\\ \hline
	47. & Viele IT-Systeme müssen miteinander Verknüpft werden.
	&
	\\ 
	
\end{longtable}