% !TEX root = Herausforderungen.tex

\chapter{Herausforderungen}

In Kapitel \ref{Kap:Analyse} wurden Herausforderungen und Metainformationen der jeweiligen Unternehmenstypen ermittelt. In diesem Kapitel werden alle ermittelten Herausforderungen hinsichtlich der Unternehmenstypen Startup, Mittelstand und Konzern betrachtet. Die Herausforderungen werden jeweils bewertet, wie Schwerwiegend diese f�r die agile Transformation des jeweiligen Unternehmenstyps sind. Nachdem ermittelt wurde, was die schwerwiegendsten Herausforderungen sind, werden Ratschl�ge zur Beseitigung dieser aufgef�hrt. Ziel dieses Kapitels ist es, dem Leser zu vermitteln, was genau Unternehmen davon abh�lt, eine agile Transformation durchzuf�hren. 

\section{Zuordnung nach Unternehmenstyp} \label{Kap:ZuordnungSchwierigkeiten}
Im ersten Schritt werden alle ermittelten Herausforderungen (siehe Kapitel \ref{Kap:Ergebnisse}) aus der Sicht des jeweiligen Unternehmenstyps betrachtet. Dabei wird entschieden, ob die Herausforderung �berhaupt auf den Unternehmenstyp zutrifft und wie Schwerwiegend die Auswirkungen auf die agile Transformation sind. Die ebenfalls in Kapitel \ref{Kap:Ergebnisse} ermittelten Metainformationen werden zus�tzlich zur Betrachtung verwendet.
\\\\
TODO: HIER BEWERTUNGSSYSTEM BESCHREIBEN!

\subsection{Startup}

\subsection{Mittelstand}

\subsection{Konzern}
Welche Herausforderungen haben welche Unternehmenstypen und wie Schwerwiegend sind diese f�r die Transformation?
Ampelsystem? Zahlensystem?

\section{Sortierung / Interpretation?} \label{Kap:Sortierung}
Was sind die Schwerwiegendsten Herausforderungen f�r den jeweiligen Unternehmenstyp? Was genau h�lt die verschiedenen Unternehmenstypen davon ab, eine agile Transformation durchzuf�hren?

\section{Ratschl�ge zur Beseitigung der Herausforderungen} \label{Kap:Ratschlaege}
Wie k�nnen die Herausforderungen gemindert oder gar komplett vermieden werden?

