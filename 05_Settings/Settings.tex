% !TEX root = Settings.tex

%################################################################## COMMANDS
%##################################################################
%################################################################## 

\newcommand{\backsl}{\verb|\|} % Backslash

\newcommand{\secline}{\vspace{-1.2em}	% Linie
	\par\noindent\rule{\textwidth}{0.4pt}} 

\newcommand{\emptysite}{\newpage	% Leere Seite
	\thispagestyle{empty}
	\section*{}}

% Glossarzusatz
\renewcommand{\glossarysection}[2][]{\chapter*{#1}}	

% Abbildungsverzeichnis
\newcommand{\abbildungsverzeichnis}{\newpage
	\phantomsection
	\addstarredchapter{\textbf{Abbildungsverzeichnis}}
	\listoffigures
}

% Tabellenverzeichnis
\newcommand{\tabellenverzeichnis}{\newpage
	\phantomsection
	\addstarredchapter{\textbf{Tabellenverzeichnis}}
	\listoftables
}

% Acronyme
\newcommand{\abkuerzungen}{
	% !TEX root = Acronym.tex
\newpage
\phantomsection
\addstarredchapter{\textbf{Abk�rzungsverzeichnis}}

\chapter*{\huge\textbf{Abk�rzungsverzeichnis}}
\begin{acronym}[Bash]
	\acro{BDD}{Behavior Driven Development}
	\acro{GUI}{Graphical User Interface}
	\acro{HTML}{Hypertext Markup Language}
	\acro{IDE}{Integrated Development Environment}
	\acro{QA}{Quality Assurance}
	\acro{QM}{Qualit�tsmanagement}	
\end{acronym}
}

% Glossar
\newcommand{\glossar}{
	\newpage
	\phantomsection	
	\addstarredchapter{\textbf{Glossar}} % F�gt "Glossar" zum Inhaltsverzeichnis hinzu		
	%\addcontentsline{toc}{chapter}{\textbf{Glossar}}	% F�gt "Glossar" zum Inhaltsverzeichnis hinzu
	\printglossary[style=altlist,title=Glossar]		% Glossar
	\newpage
}

% Heading Style setzen
\newcommand{\headstyle}{
	\pagestyle{scrheadings}	% Style
	\clearscrheadfoot
	\ofoot[\pagemark]{\pagemark}
	\ohead{\headmark}
	\automark[]{chapter}
	\setheadsepline{1pt}
	\setfootsepline{0pt}
}

% Heading Style setzen (Seitenzahl Rechts)
\newcommand{\headstyleright}{
	\pagestyle{scrheadings}	% Style
	\cfoot[]{}	% [plain-Seiten]{normale Seiten} 
	\ofoot[\pagemark]{\pagemark}	% Seitenzahl
	\setheadsepline{0pt}
	\setfootsepline{0pt}
}

\newcommand{\inhaltsverzeichnis}{
	\cleardoublepage\pdfbookmark{\contentsname}{toc}
	\tableofcontents
	\clearpage
	\newpage
}

\newcommand{\literaturverzeichnis}{
	\phantomsection
	\addcontentsline{toc}{chapter}{Literatur}	% �berschrift
	\printbibliography	
}

\newcommand{\anhangstart}{
	\newpage
	%\pagenumbering{arabic}	% Arabische Ziffern
	
	\renewcommand\appendix{\par 
		\setcounter{section}{0}% 
		\setcounter{subsection}{0}% 
		\setcounter{figure}{0}% 
		
		\renewcommand\thesection{\Alph{section}}% 
		\renewcommand\thefigure{\Alph{section}\arabic{figure}}} 
	
	\numberwithin{table}{section} 
	\numberwithin{figure}{section} 
	
	
	\appendix 
	\pagestyle{empty}
	\captionsetup{listof=false}
}

\newcommand{\roemischenummerierung}{
	\pagenumbering{Roman}
}

\newcommand{\arabischenummerierung}{
	\pagenumbering{arabic}
}

%################################################################## OTHER
%##################################################################
%################################################################## 

\titleformat{name=\chapter,numberless}[display]{\normalfont\huge\bfseries}{\titlerule}{-6,9ex}{}[\vspace{3ex}]
\titlespacing{name=\chapter,numberless}{0em}{1.0em}{0em}
\titleformat{\chapter}{\normalfont\huge\bfseries}{\thechapter}{18pt}{\Huge}[{\vspace{0,6ex}\titlerule[0.8pt]\vspace{3ex}}]
\titlespacing{\chapter}{0em}{-4.0em}{0em} % {left}{before}{after}[right]
\titleformat{\section}{\normalfont\LARGE\bfseries}{\thesection}{16pt}{\LARGE}
\titleformat{\subsection}{\normalfont\Large\bfseries}{\thesubsection}{14pt}{\Large}
\titleformat{\subsubsection}{\normalfont\large\bfseries}{\thesubsubsection}{14pt}{\large}

\addbibresource{04_Literatur/Literatur.bib}
\AtBeginDocument{\counterwithin{lstlisting}{section}}
\geometry{a4paper, left=30mm, right=25mm, bottom=25mm, top=25mm}	% Text Formatierung
% !TEX root = Glossar.tex
\makeglossaries

\newglossaryentry{venturecapital}{ 
	name=Venture-Capital, 
	description={ 
		Unter Venture Capital versteht man die Finanzierung eines nicht b�rsennotierten Unternehmens mit Eigenkapital. Dabei wird das Unternehmen in der Regel zum Zeitpunkt der Beteiligung privat gehalten. Entscheidend ist, dass das Eigentum am Unternehmen w�hrend der meisten Zeit der Beteiligung in privaten H�nden liegt \autocite{venture_capital}} 
}

\newglossaryentry{servantleader}{ 
	name=Servant Leader, 
	description={ 
		Ein Servant Leader liebt Menschen und m�chte ihnen helfen. Die Aufgabe eines Servant Leaders ist es daher, die Bed�rfnisse anderer zu ermitteln und zu versuchen, diese Bed�rfnisse zu befriedigen \autocite{servant_leadership}} 
} 

\newglossaryentry{iteration}{ 
	name=Iteration, 
	description={ 
		Eine Iteration ist eine wiederholte Durchf�hrung eines Vorgangs. Die Anzahl der Durchf�hrungen (Iterationen) steht entweder vorher fest oder richtet sich nach der Erf�llung eines Abbruchkriteriums \autocite{definition_iteration} 
	}
} 

\newglossaryentry{stakeholder}{ 
	name=Stakeholder, 
	description={ 
		Anspruchsgruppen sind alle internen und externen Personengruppen, die von den unternehmerischen T�tigkeiten gegenw�rtig oder in Zukunft direkt oder indirekt betroffen sind. Gem�� Stakeholder-Ansatz wird ihnen - zus�tzlich zu den Eigent�mern (Shareholders) - das Recht zugesprochen, ihre Interessen gegen�ber der Unternehmung geltend zu machen \autocite{definition_stakeholder}
	}
}

\newglossaryentry{trialanderror}{ 
	name=Trial and Error, 
	description={ 
		Trial and Error beschreibt einen Weg, ein Ziel zu erreichen oder ein Problem zu l�sen, indem verschiedene Methoden ausprobiert und aus Fehlern gelernt wird \autocite{definition_trialanderror}
	}
}

\newglossaryentry{governance}{ 
	name=Governance, 
	description={ 
		Die Art und Weise, wie in einem Land die Angelegenheiten der Allgemeinheit verwaltet und geregelt werden. \enquote{Governance} ist nicht \enquote{Government} gleich \enquote{Regierung}, sondern umfassender und offener \autocite{definition_governance}
	}
}

\newglossaryentry{workaround}{ 
	name=Workaround, 
	description={ 
		Ein workaround beschreibt eine M�glichkeit, ein Problem zu l�sen, oder etwas trotz des Problems zum laufen zu bekommen, ohne es vollst�ndig zu l�sen \autocite{definition_workaround}
	}
}

\newglossaryentry{coredeliverable}{ 
	name=Core Deliverable, 
	description={ 
		Ein Core Deliverable ist das zentrale Produkt oder Dienstleistung, welches einem Kunden zu Verf�gung gestellt wird \autocite{definition_deliverable}
	}
}

\newglossaryentry{cicd}{ 
	name=Continous Integration, 
	description={ 
		Continous Integration beschreibt einen Softwareentwicklungsprozess, bei dem Teammitglieder h�ufig ihre Arbeit integrieren. Bei jeder Integration wird der Stand automatisch �berpr�ft, um Integrationsfehler so schnell wie m�glich zu erkennen \autocite{definition_ci}
	}
}

\newglossaryentry{devops}{ 
	name=DevOps, 
	description={ 
		Der Begriff besteht aus Development (Dev), der den Softwareentwickler beschreibt, und Operations (Ops), was den IT-Betrieb darstellt. Aus der Kombination beider soll ein Prozessverbesserungsansatz f�r die Bereiche Softwareentwicklung und Systemadministration entstehen. Es wird eine effektivere und effizientere Zusammenarbeit der verschiedenen Bereiche angestrebt \autocite{definition_devops}
	}
}
	% Glossar IMPORT
\setlength{\parindent}{0pt}	% Keine Einr�ckung am Anfang eines Absatzes
\setcounter{biburllcpenalty}{7000}
\setcounter{biburlucpenalty}{8000}

%################################################################## LISTINGS
%##################################################################
%################################################################## 

\renewcommand{\lstlistingname}{Codebeispiel}% Listing -> Codebeispiel
\renewcommand{\lstlistlistingname}{Codeverzeichnis}% List of Listings -> Codeverzeichnis

% Default fixed font does not support bold face
\DeclareFixedFont{\ttb}{T1}{txtt}{bx}{n}{12} % for bold
\DeclareFixedFont{\ttm}{T1}{txtt}{m}{n}{12}  % for normal

% Custom colors
\definecolor{deepblue}{rgb}{0,0,0.5}
\definecolor{deepred}{rgb}{0.6,0,0}
\definecolor{deepgreen}{rgb}{0,0.5,0}
\definecolor{superlightgray}{rgb}{0.97,0.97,0.97}

% Java Style
\lstdefinestyle{java}{
	language=Java,
	basicstyle=\ttm,
	otherkeywords={this},
	breaklines=true,
	tabsize=2,
	keywordstyle=\ttb\color{deepblue},
	emph={MyClass,__init__},          % Custom highlighting
	emphstyle=\ttb\color{deepred},    % Custom highlighting style
	stringstyle=\color{deepgreen},
	frame=tb,                         % Any extra options here
	belowskip=3em,
	belowcaptionskip=1em,
	aboveskip=2em,
	abovecaptionskip=1em,
	captionpos=b,
	showstringspaces=false 
}

% Java Script Style
\lstdefinestyle{travis}{
	language=bash,
	numbers=left,
	stepnumber=5,
	firstnumber=1,
	tabsize=2,
	basicstyle=\ttm,
	breaklines=true,
	alsoletter={:},
	keywords={language:, node_js:, cache:, install:, before_script:, script:},
	keywordstyle=\ttb\color{orange},
	emph={MyClass,__init__},          % Custom highlighting
	emphstyle=\ttb\color{deepred},    % Custom highlighting style
	stringstyle=\color{deepgreen},
	frame=single,                         % Any extra options here
	backgroundcolor=\color{superlightgray},
	ndkeywords={directories:},
	ndkeywordstyle=\ttb\color{deepblue},
	belowskip=2em,
	aboveskip=2em,
	abovecaptionskip=1em,
	captionpos=b,
	showstringspaces=false 
}

% Java Script Style
\lstdefinestyle{cypress}{
	language=Java,
	numbers=left,
	stepnumber=5,
	firstnumber=1,
	tabsize=2,
	basicstyle=\ttm,
	breaklines=true,
	keywords={describe, before, beforeEach, context, it, expect, equal, cy, def},
	keywordstyle=\ttb\color{orange},
	emph={MyClass,__init__},          % Custom highlighting
	emphstyle=\ttb\color{deepred},    % Custom highlighting style
	stringstyle=\color{deepgreen},
	frame=single,                         % Any extra options here
	backgroundcolor=\color{superlightgray},
	ndkeywords={assert, break, case, catch, continue, debugger, default, delete, do, else, false, finally, for, function, if, in, instanceof, new, null, return, switch, this, throw, true, try, typeof, var, void, while, with},
	ndkeywordstyle=\ttb\color{deepblue},
	belowskip=2em,
	aboveskip=2em,
	abovecaptionskip=1em,
	captionpos=b,
	showstringspaces=false 
}

% Python Style
\lstdefinestyle{python}{
	language=python,
	numbers=left,
	basicstyle=\ttm,
	otherkeywords={self},
	keywordstyle=\ttb\color{deepblue},
	emph={MyClass,__init__},          % Custom highlighting
	emphstyle=\ttb\color{deepred},    % Custom highlighting style
	stringstyle=\color{deepgreen},
	frame=tb,                         % Any extra options here
	belowskip=3em,
	aboveskip=2em,
	abovecaptionskip=1em,
	captionpos=b,
	showstringspaces=false  
}

% Bash Style
\lstdefinestyle{bash}{
	language=bash,
	basicstyle=\ttm,
	numbers=left,
	stepnumber=5,
	firstnumber=1,
	backgroundcolor=\color{superlightgray},
	otherkeywords={},
	keywordstyle=\ttb\color{deepblue},
	emph={\$},          % Custom highlighting
	emphstyle=\ttb\color{deepred},    % Custom highlighting style
	stringstyle=\color{deepgreen},
	frame=single,                         % Any extra options here
	showstringspaces=false,
	belowskip=2em,
	aboveskip=2em,
	abovecaptionskip=1em,
	captionpos=b
}

%################################################################## DEFINITONS
%##################################################################
%################################################################## 

\newtcbtheorem[number within=chapter]{Definition}{}{
	enhanced,
	attach boxed title to top left={
		xshift=-1mm,
		yshift=-5mm,
		yshifttext=-1mm
	},
	breakable,
	top=0.5em,
	bottom=-0.5em,
	colback=white,
	colframe=black,
	fonttitle=\bfseries,
	boxed title style={
		size=small,
		colback=black,
		colframe=black,
	} 
}{def}

\newenvironment{definition}[2]{ \begin{Definition}[adjusted title=#1]{}{#2} 
		\textbf{} }{\end{Definition}}